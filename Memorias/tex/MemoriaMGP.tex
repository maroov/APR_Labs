% Preamble
% ---
\documentclass[a4paper]{article}

% Packages
% ---
\usepackage{bm}
\usepackage[spanish,es-nodecimaldot]{babel}
\usepackage[utf8]{inputenc}
\usepackage[T1]{fontenc}
\usepackage{parskip}
\usepackage{fancyhdr}
\usepackage{mathtools}
\usepackage{amsmath}
\usepackage[htt]{hyphenat}
\usepackage{capt-of}
\usepackage{graphicx}
\graphicspath{ {./images/} }
\usepackage{xcolor}
\usepackage{listings}

% Coding sections
% ---
\definecolor{codegreen}{rgb}{0,0.6,0}
\definecolor{codegray}{rgb}{0.5,0.5,0.5}
\definecolor{codepurple}{rgb}{0.58,0,0.82}
\definecolor{backcolour}{rgb}{0.95,0.95,0.92}
 
\lstdefinestyle{mystyle}{
    backgroundcolor=\color{backcolour},   
    commentstyle=\color{codegreen},
    keywordstyle=\color{magenta},
    numberstyle=\tiny\color{codegray},
    stringstyle=\color{codepurple},
    basicstyle=\ttfamily\footnotesize,
    breakatwhitespace=false,         
    breaklines=true,                 
    captionpos=b,                    
    keepspaces=true,                 
    numbers=left,                    
    numbersep=5pt,                  
    showspaces=false,                
    showstringspaces=false,
    showtabs=false,                  
    tabsize=2
}
 
\lstset{style=mystyle}

% Pagestyles
% ---
\pagestyle{fancy}
\rhead{Roselló Beneitez, N. U.; Roselló Oviedo, M.}
\lhead{APR: Práctica sobre MGP}
\fancyfoot[C]{\thepage}

% Main
% ---
\begin{document}

\author{Roselló Beneitez, N. U.; Roselló Oviedo, M.}
\title{APR: Práctica sobre Modelos Gráficos Probabilísticos}
\date{6 de Enero de 2020}
\maketitle{}
\thispagestyle{empty}

\newpage
\tableofcontents
\listoffigures

\newpage
\section{Descripción de la práctica}
\quad 

\section{Ejercicio A}
\quad Comparemos los resultados para los datos completos e incompletos:

    \begin{table*}[h!]
      \noindent \begin{minipage}{.45\columnwidth}
        \begin{lstlisting}[language=octave]
		Datos completos
		W:
		1 1 : 1.0000 0.0000 
		2 1 : 0.0556 0.9444 
		1 2 : 0.0435 0.9565 
		2 2 : 0.0000 1.0000 
		S:
		1 : 0.5532 0.4468 
		2 : 0.9057 0.0943 
		R:
		1 : 0.7234 0.2766 
		2 : 0.2264 0.7736 
		C:
		1 : 0.4700 
		2 : 0.5300 
		\end{lstlisting}
      \end{minipage}\hfill
      \begin{minipage}{.6\columnwidth}
      	\begin{lstlisting}[language=octave, xleftmargin=1cm, xrightmargin=1cm]
		Datos incompletos
		W:
		1 1 : 0.9998 0.0002 
		2 1 : 0.0136 0.9864 
		1 2 : 0.1177 0.8823 
		2 2 : 0.0048 0.9952 
		S:
		1 : 0.5556 0.4444 
		2 : 0.9999 0.0001 
		R:
		1 : 0.5518 0.4482 
		2 : 0.2445 0.7555 
		C:
		1 : 0.5425 
		2 : 0.4575 
		\end{lstlisting}
      \end{minipage}
    \end{table*}

\section{Ejercicio B}
\quad El script \textit{matlab} para la red de diagnóstico de cáncer de pulmón se adjuntará, comentado y listo para su ejecución, junto con esta memoria. Mediante su utilización, se ha podido responder a las siguientes cuestiones:

\begin{itemize} 
\item ¿Cuál es la probabilidad de que un paciente no fumador no tenga cáncer de pulmón si la radiografía ha dado un resultado negativo pero sufre disnea?
\[ P \left( !C | R = n, D = s \right) = 0.9949 = 99.49\% \]
\item ¿Cuál es la explicación más probable de que un paciente sufra cáncer de pulmón?
\[ \lbrace \left( 1 \right) \rbrace \quad \lbrace \left( 2 \right) \rbrace \quad \lbrace \left( 2 \right) \rbrace \quad \lbrace \left( 3 \right) \rbrace \quad \lbrace \left( 2 \right) \rbrace \]
Lo cual se traduce en polución \textbf{baja}, fumador \textit{sí}, cáncer \textbf{positivo}, rayos X \textbf{positivo} y disnea \textbf{sí}.

El log-verosimilitud de esta explicación es de $-5.0925$, esto es, una probabilidad del $0.61\%$.
\end{itemize}

\section{Conclusiones}
\quad 

\end{document}
